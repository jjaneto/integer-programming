\documentclass[12pt]{article}
\usepackage[utf8]{inputenc}
\usepackage{amsmath}
\usepackage{geometry}
 \geometry{
 a4paper,
 left=30mm,
 top=30mm,
 right=20mm,
 bottom=20mm,
 }

\usepackage{pslatex}
\usepackage{setspace} % espacamento entre linhas

% padrao 1.5 de espacamento entre linhas
\setstretch{1.5}

\title{Programação Inteira - 2IS}
\author{José Joaquim de Andrade Neto}
\date{February 2018}

\begin{document}

\maketitle

\section{Conjunto Independente de Vértices}

Dado um grafo simples $G(V, A)$ um \textbf{Conjunto Independente de Vértices} - \textit{Independent Set} - é um subconjunto $X \subseteq V$ de vértices tal que para todo par de vértice $u, v \in X$ não há uma aresta $(u, v)$ no grafo original. O problema do \textbf{Conjunto Máximo Independente de Vértices} - \textit{Maximum Independent Set} (MIS) - é um problema que consiste em achar o conjunto máximo (isto é, que não dá para ser expandido) de vértices independentes. Esse problema é $\mathcal{NP}$-Difícil para grafos em geral, o que proporcionou diversos estudos na literatura sobre como achar tais conjuntos em um tempo viável. Tais estudos levam em consideração modelagens exatas ou heurísticas, na tentativa de geração de soluções.

Uma formulação inteira para o MIS é denotada por IP:MIS e é computada por

\begin{align*}	
  \max\quad        & \sum_{u \in V} u_i                                                          \\
  \text{s.a.\quad} & u_i + u_j \leq 1 &  \forall (u, v) \in A  								 \\
  				   & u_i \in \{0, 1\} & \forall u \in V,
\end{align*}

\noindent
onde $u_i$ tem valor $1$ se e somente se ele pertence ao conjunto independente.

O problema de achar \textsl{dois} conjuntos independentes de vértices é definido de forma similar: a solução gerada deve retornar dois conjuntos independentes e disjuntos. Essa variante possui modelagem similar, uma vez que o método usado para gerar a solução é bastante semelhante. Uma vez que um conjunto foi gerado, exclui-se os vértices que o compõe, e tenta-se achar um novo conjunto com o novo grafo, gerado a partir da exclusão dos vértices. Ao final, se o grafo for viável, haverão dois conjuntos independentes.


\section{Experimentos}
\subsection{Modelagem Básica}
\subsection{Rotina CLQ1}
\subsection{Branch-and-Cut}
\subsection{Heurísticas}




\end{document}
