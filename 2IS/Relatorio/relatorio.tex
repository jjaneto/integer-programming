\documentclass[12pt]{article}
\usepackage[utf8]{inputenc}
\usepackage{amsmath}
\usepackage{geometry}
\geometry{
  a4paper,
  left=30mm,
  top=30mm,
  right=20mm,
  bottom=20mm,
}

\usepackage{pslatex}
\usepackage{setspace}
\usepackage{amssymb}
\usepackage{minted}

\setstretch{1.5}

\title{Programação Inteira - 2IS}
\author{José Joaquim de Andrade Neto}
\date{February 2018}

\begin{document}

\maketitle

\section{O problema do 2 Conjuntos Independentes de Vértices}

Dado um grafo simples $G(V, A)$, onde $V$ é o conjunto de vértices e $A$ é o conjunto de arestas, define-se o problema de 2 Cojuntos Independentes de Vértices (2MIS) como sendo o problema de determinar dois conjuntos independentes de vértices em $G$, tal que a soma desses dois conjuntos seja máxima e os conjuntos sejam distintos.
Cada conjunto independente é, na verdade, um subconjunto $B \subseteq V$ tal que não existe aresta $(u, v)$ que liga os vértices $u$ e $v$. Em outras palavras, $B$ é um subconjunto de vértices não adjacentes. Para serem considerados distintos, dois conjuntos independentes não devem possuir o mesmo vértice pertencente ao mesmo tempo nos dois conjuntos. Isso significa que, dado um segundo conjunto independente $D$, $B \cap D = \emptyset$.

O restante desse trabalho irá apresentar uma formulação inteira para o 2MIS, na qual uma irá trabalhar sob as arestas de $G$, enquanto que a outra trabalha baseada nas cliques maximais de $G$. Ainda, para a obtenção de soluções de forma mais eficiente, e que essas soluções sejam primais, também são apresentadas duas heurísticas. Todas essas definições foram implementadas nos experimentos.  

\subsection{Formulação Original}

O problema é caracterizado por definir dois conjuntos independentes, $\alpha$ e $\beta$, pertencentes ao grafo $G(V, A)$, tais que sejam distintos e máximos. Ele pode ser definido da seguinte forma:

\begin{equation*}
\begin{array}{ll@{}ll}
\text{maximize}  & \displaystyle(\sum\limits_{i \in \alpha} x_{i}\ +\ \sum\limits_{j \in \beta}y_j)\\
\text{sujeito a }& \displaystyle x_u + x_v \leq 1 &\forall \ (x_u, x_v) \in A & (1) \\
                 & \displaystyle y_j + y_i \leq 1 &\forall \ (y_j, y_i) \in A & (2) \\
                 & \displaystyle x_u + y_u \leq 1 &\forall \ u \in \alpha \wedge u \in \beta & (3)\\
                 & \displaystyle x_i \wedge y_j \in \{0,1\} &\forall \ i \in \alpha \wedge \forall \ j \in \beta & (4),
\end{array}
\end{equation*}

\noindent
onde $x_i = 1$ ou $y_j = 1$ somente se $i$ ou $j$ fizerem parte de seus respectivos conjuntos independentes, $\alpha$ e $\beta$ ($4$). As restrições $(1)$ e $(2)$ garantem que os dois conjuntos são independentes, enquanto que a $(3)$ assegura que os conjuntos são distintos, isto é, dado um vértice $u \in V$, então ele estará no máximo em um dos dois conjuntos ($x_u + y_u \leq 1$).

\subsection{Formulação com Cliques}

É possível formular o problema do 2MIS usando cliques. Uma clique é um subconjunto de vértices $X \subseteq V$ tal que todos os pares de vértices são ligados entre si. Em outras palavras, o grafo induzido por $X$ é completo. Um conjunto independente de vértice não contém mais de um vértice em cada clique. Portanto, dado uma clique $C$, têm-se que $\sum_{u \in C}x_u \leq 1$ para todas as cliques $C$ pertencentes a $G$. Adaptando essa formulação para o problema do 2MIS:

\begin{equation*}
\begin{array}{ll@{}ll}
\text{maximize}  & \displaystyle(\sum\limits_{i \in \alpha} x_{i}\ +\ \sum\limits_{j \in \beta}y_j)\\
\text{sujeito a }& \displaystyle \sum\limits_{i \in C} x_i \leq 1 &\forall \ cliques \ C \in \alpha & (1) \\
                 & \displaystyle \sum\limits_{j \in C} x_j \leq 1 &\forall \ cliques \ C \in \beta & (2) \\
                 & \displaystyle x_u + y_u \leq 1 &\forall \ u \in \alpha \wedge u \in \beta & (3)\\
                 & \displaystyle x_i \wedge y_j \in \{0,1\} &\forall \ i \in \alpha \wedge \forall \ j \in \beta & (4),
\end{array}
\end{equation*}

\section{Heurísticas}

Heurísticas permitem que sejam obtidas soluções primais a partir do arredondamento de variáveis fracionárias de soluções lineares ótimas. O arredondamento dá-se a partir da escolha de um vértice $v$ com $x_v$ fracional e atribuindo o valor $x_v = 1$ para ele e $x_u = 0$ para todos os seus adjacentes. Esse processo é executado até que $x_v = \{0, 1\}$ para todos os $v$ dos dois conjuntos. Duas heurísticas são aplicadas, descritas a seguir:

\begin{description}
\item[RND1]: selecione uma variável que maximiza $\{x_v : x_v < 1\}$
\item[RND2]: selecione uma variável que minimiza $\{x_v + \sum\limits_{(x_v, x_u) \in A}x_u : 0 < x_v < 1\}$
\end{description}

Após as duas heurísticas serem aplicadas, suas soluções são comparadas, e a que maximiza a função objetivo é escolhida.

\section{Implementação}

Todos os códigos foram implementados em $C++11$ e possuem um esquema padrão para as suas respectivas execuções. Para compilar, basta inserir o comando \textsl{make} no diretório dos códigos-fonte. Após a compilação, basta digitar o modo a ser testado (formulação original, com clique, com ou sem heurísticas, etc) através dos argumentos e o caminho para a instância, a qual já está dentro do diretório. Os argumentos para uma formulação deve receber, em ordem, se usará heurísticas ($h$) ou não ($nh$), e se fará o uso de cortes ($c$) ou não ($nc$). Um exemplo que usa heurística mas não corte é descrito a seguir:

\begin{minted}{c}
  ./2clq1_formulation_c++ h nc ./instancias g10_5_1
\end{minted}

Os algoritmos são divididos em fases, sendo estas (i) de leitura do grafo, (ii) pré-processamento (no caso das cliques) e (iii) execução e resultado, ilustrando os dois conjuntos independentes encontrados. A leitura é realizada na função \textsl{readgraph()}, a qual redireciona \textsl{STDIN} para o arquivo da instância e lê a lista de adjacência do grafo. O pré-processamento trata de computar as cliques (no caso da CLQ1, na função \textsl{clq1()}, e adicionar os parâmetros do problema inteiro (restrições, função objetivo, etc).

A execução acontece na função \textsl{runOptimization()}. Por fim, a saída do Gurobi também é codificada nessa mesma função. O \textit{Callback} é implementado na classe \textsl{mycallback}. Nessa classe estão implementadas as heurísticas e os cortes utilizado nos experimentos. Considerando o diretório dos códigos fontes como sendo \textsl{./}, então os resultados (a saída do Gurobi) estão no diretório \textsl{./results-out}.

\section{Resultados}

Os resultados gerados nesse trabalho abordam somente aqueles executados sem heurísticas e sem cortes, uma vez que nas instâncias pequenas não houveram diferenças entre os resultados. Não há resultados obtidos a partir das instâncias grandes, uma vez que elas demoraram para executar e nesse caso foram interrompidas.



\end{document}
